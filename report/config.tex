% Basics
% ===========================================================
\usepackage[english]{babel}
\usepackage[T1]{fontenc}
\usepackage{csquotes}
\usepackage{etex}
\usepackage{xcolor}
\usepackage{xspace}
\usepackage{etoolbox}
\usepackage{xparse}
\usepackage{ifthen}
\usepackage{calc}
% ===========================================================

% graphics, documents, links
% ===========================================================
\usepackage{graphicx}
\usepackage{graphics}
\usepackage{svg}
\usepackage{pdfpages}
\usepackage{hyperref}
\usepackage{epstopdf}
\AppendGraphicsExtensions{.pdf}


% ===========================================================
% Figures
% ===========================================================
\usepackage{float}
\NewDocumentCommand{\Fig}{ O{1} O{H} m m m }{%
	\begin{figure}[#2]
		\centering
		\includegraphics[width=#1\linewidth]{#3}%
		\caption{#4}%
		\label{fig:#5}%
	\end{figure}%
}



% Fonts & Typography
% ===========================================================
\linespread{1.25}
\usepackage{sourcecodepro}
\usepackage[default]{sourcesanspro}
\usepackage{amsfonts}
\usepackage{stmaryrd}
\usepackage{bbm}
\usepackage{bm}
\newcommand{\code}[1]{\texttt{#1}}
\usepackage{xspace}
\makeatletter 
\xspaceaddexceptions{\grqq \grq \csq@qclose@i \} } 
\makeatother
\usepackage{multicol}
\usepackage{vwcol}
\usepackage{soul}
\usepackage{eurosym}
% ===========================================================


% Mathe-Pakete
% ===========================================================
\usepackage{mathtools}
\usepackage{amssymb}
\usepackage[bigdelims]{newtxmath}
\usepackage{wasysym}
% ===========================================================

% TikZ -- hope we dont need it
% ===========================================================
%\usepackage{tikz}
%\usepackage{tikz-cd}          
%\usetikzlibrary{arrows.meta}    
%\usetikzlibrary{shadows}
%\usetikzlibrary{calc}
%\usetikzlibrary{backgrounds}  
%\usetikzlibrary{patterns}
%\tikzset{>=Latex}

%\usepackage{tikz-qtree}
%\usetikzlibrary{positioning, shapes.geometric}
% ===========================================================

% pgfplots - dont needed
% ===========================================================
% \usepackage{pgfplots}
% \pgfplotsset{width=10cm,compat=1.9}
% ===========================================================

% Tabel
% ===========================================================
\usepackage{tabularray} % best tables
% ===========================================================

% listings
% ===========================================================
\usepackage{listingsutf8}
\usepackage[most]{tcolorbox}
% \begin{lstlisting}[style=pseudo]
\lstdefinestyle{pseudo}{
	belowcaptionskip=1\baselineskip,
	breaklines=true,
	showstringspaces=false,
	keepspaces,
	showstringspaces=false,
	basicstyle=\small,
	keywordstyle=\bfseries,
	commentstyle=\itshape\color{gray},
	numbers=left,
	numberstyle=\tiny\ttfamily\color{darkgray},
	inputencoding=utf8/latin1,
	tabsize=2,
	mathescape,
	morekeywords={if, then, return, end, function, for, to, do, while, else, and, procedure},
	literate={:=}{$\leftarrow$}1,
	moredelim=[is][\normalfont\raggedleft]{|}{|}
}

% really complex stuff for nice python code snippets
\newlength{\LstNumField}
\setlength{\LstNumField}{40em}
\makeatletter
\renewcommand{\thelstnumber}{\ifnum\value{lstnumber}<10 0\fi\arabic{lstnumber}}
\makeatother
\makeatletter
\newcommand{\zebra}[2]{%
	\begingroup
	\lst@basicstyle
	\ifodd\value{lstnumber}%
	\color{#1}%
	\else
	\color{#2}%
	\fi
	\rlap{\hspace*{\lst@numbersep}%
		\color@block{1.015\linewidth}{\ht\strutbox}{\dp\strutbox}%
	}	
	\endgroup 	
}
\makeatother

\lstdefinestyle{python}{
	language=Python,
	%belowcaptionskip=1\baselineskip,
	breaklines=false,
	showstringspaces=false,
	keepspaces,
	basicstyle=\small\ttfamily,
	keywordstyle=\bfseries\color{blue},
	commentstyle=\itshape\color{green},
	stringstyle=\color{orange},
	numbers=left,
	inputencoding=utf8,
	tabsize=2,
	numberstyle=\ttfamily\zebra{gray!15}{gray!2},
	numbersep=2.2em}

\DeclareRobustCommand{\PythonFile}[3]{
	\begin{tcolorbox}[colback=gray!15, colframe=black, boxrule=0.5pt, arc=1pt, left=16.5pt, right=-9.5pt, top=-5pt, bottom=-5pt,boxsep=0pt]
			\lstinputlisting[
			style=python,
			firstline=#2,
			lastline=#3,
			firstnumber=#2
			]{#1}
		\end{tcolorbox}
		\vspace{-4pt}
		{\tiny #1: Line #2-#3 \medskip\par}
}
% ===========================================================

% Bibliography for References
% ===========================================================
  \usepackage[
	  backend=biber,
	  style=numeric,
	  sorting=ynt,
	  citestyle=numeric
  ]{biblatex}
  \addbibresource{./modules/references.bib}
  \renewcommand*{\mkbibbrackets}[1]{\textsuperscript{[#1]}}
  \nocite{*}    

% Hyperlinks within Document and urls
\hypersetup{
	colorlinks=true,
	linkcolor=black,
	filecolor=magenta,      
	urlcolor=cyan,
	pdfpagemode=FullScreen,
	citecolor=blue
}
\urlstyle{same}

% ===========================================================



% table of contents
% ===========================================================
\newcommand{\toc}{
		\tableofcontents
		\pagebreak
}
% ===========================================================

% Empty Line
% ===========================================================
\newcommand{\blank}{\phantom{ }\\}
% ===========================================================


% Quotes
% ===========================================================
\newcommand{\qt}[1]{\glqq #1\grqq \xspace }
% ===========================================================

% Shortcuts
% ===========================================================
\newcommand{\B}{\mathbb{B}}
\newcommand{\C}{\mathbb{C}}
\newcommand{\N}{\mathbb{N}}
\newcommand{\Q}{\mathbb{Q}}
\newcommand{\R}{\mathbb{R}}
\newcommand{\Z}{\mathbb{Z}}
\newcommand{\oh}{\mathcal{O}}

\newcommand{\ra}{\rightarrow}
\newcommand{\imp}{\Rightarrow}
\newcommand{\limp}{\Leftarrow}
\newcommand{\eq}{\Leftrightarrow}


\newcommand{\ol}[1]{\overline{#1}}
\newcommand{\wt}[1]{\widetilde{#1}}
\newcommand{\wh}[1]{\widehat{#1}}

\DeclareMathOperator{\id}{id}         % Identity
\DeclareMathOperator{\pot}{\mathcal{P}}    % Potency Set
% ===========================================================

% Brackets
% ===========================================================
\DeclarePairedDelimiter{\absolut}{\lvert}{\rvert}    % Abs
\DeclarePairedDelimiter{\ceiling}{\lceil}{\rceil}    % Ceil
\DeclarePairedDelimiter{\Floor}{\lfloor}{\rfloor}    % Floor
\DeclarePairedDelimiter{\Norm}{\lVert}{\rVert}      % Norm
\DeclarePairedDelimiter{\sprod}{\langle}{\rangle}    
\DeclarePairedDelimiter{\enbrace}{(}{)}          % n
\DeclarePairedDelimiter{\benbrace}{\lbrack}{\rbrack}  
\DeclarePairedDelimiter{\penbrace}{\{}{\}}        
\newcommand{\Underbrace}[2]{{\underbrace{#1}_{#2}}}   

\newcommand{\abs}[1]{\absolut*{#1}}
\newcommand{\ceil}[1]{\ceiling*{#1}}
\newcommand{\flo}[1]{\Floor*{#1}}
\newcommand{\no}[1]{\Norm*{#1}}
\newcommand{\sk}[1]{\sprod*{#1}}
\newcommand{\enb}[1]{\enbrace*{#1}}
\newcommand{\penb}[1]{\penbrace*{#1}}
\newcommand{\benb}[1]{\benbrace*{#1}}

\makeatletter
\renewcommand*\env@matrix[1][*\c@MaxMatrixCols c]{%
	\hskip -\arraycolsep
	\let\@ifnextchar\new@ifnextchar
	\array{#1}}
\makeatother
% ===========================================================