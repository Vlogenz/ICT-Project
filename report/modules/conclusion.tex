
\section{Conclusion}

This report documented the design, implementation, and evaluation of \textit{CircuitQuest}, a software application developed to serve as an interactive educational tool for digital logic and computer architecture.

\subsection{Summary of Work and Results}
The objective of the project was to design and implement a gamified platform that guides students from fundamental Boolean logic through to the complex architecture of a functioning single-cycle MIPS processor. 


Key outcomes of this project include:
\begin{itemize}
    \item Functional Datapath: We successfully implemented and tested the core components required to simulate a single-cycle MIPS datapath.
    \item Robust Architecture: The adoption of the Model-View-Controller (MVC) pattern and an Event Bus mechanism ensured a decoupled system where the simulation logic remains isolated and testable, independent of the front-end visualization.
    \item Validated Logic: Through comprehensive unit testing, the logic simulation layer achieved a test coverage of 93\%, validating the correctness of critical components, which is essential for an educational tool where accuracy is paramount.
\end{itemize}

\subsection{Project Implications and Gaps}
\textit{CircuitQuest} provides a demonstrable solution to the problem of visualizing complex component interaction and control flow, concepts that are difficult to teach using static methods alone.

While the project provides a strong foundation, the current implementation contains scope limitations that warrant future attention. The most significant gap is the simplified, pre-built nature of the Control Unit and the restriction on loading user-defined instructions in the Sandbox environment. Furthermore, the reliance on Python, while enabling rapid development, imposes performance constraints that would necessitate a transition to a lower-level language for a truly commercial-grade product.

In conclusion the project delivered a dactically viable and accurate educational tool, validating the core principle that interactive simulation can significantly deepen a student's understanding of computer architecture.
