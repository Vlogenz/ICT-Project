% - Background
% - Problem definition
% - Assumptions and limitations
% - Literature study
% - 


\section{Introduction}

\subsection{Background}
During the previous semester, four members of our group attended a lecture that focused on the logical design of a MIPS processor. The course provided insight into how digital logic forms the foundation of modern computer architectures. Although the subject matter was highly engaging, the complexity of the concepts --- especially regarding component interaction and processor control flow --- proved difficult to fully comprehend using traditional teaching methods alone.
This experience motivated us to create an interactive educational platform that better supports the learning process. With \textit{CircuitQuest}, we aim to provide a tool that allows users not only to explore theoretical concepts, but also to practically build, test, and simulate custom logic components. We also see the potential for ongoing development of this project, including possible collaboration with Prof. Dr. Paula Herber at the University of Münster, whose lecture \textit{Informatik IV -- Rechnerstrukturen} covers related subject matter and could benefit from an applied learning environment such as CircuitQuest.

\subsection{Problem Definition}
The objective of this project is to design and implement a software application that teaches fundamental and advanced principles of digital logic to computer science students. CircuitQuest guides users from simple Boolean logic and basic gates through more complex modules such as multiplexers, memory elements, and arithmetic logic units (ALUs). The ultimate goal is to provide users with the ability to construct and simulate a functioning single-cycle processor, deepening their understanding of how individual components contribute to a cohesive CPU architecture.

\subsection{Assumptions and Limitations}
We assume that our target audience --- primarily computer science students --- already possesses an intuitive interest in computing and logical problem-solving.
Due to the academic scope of the course, the project was limited by the short development time and the fact that several group members learned Python during the process. Coordinating work among an unusually large team also introduced organisational challenges; however, this collaborative structure enabled us to undertake a more ambitious and feature-rich project than would normally be possible.

\subsection{Technology Choices}
CircuitQuest is developed using Python, with the graphical user interface implemented through the Qt framework. This technology choice supports interactive, cross-platform development and allowed us to quickly prototype and refine user-focused features.
Given the size of the group and the distributed nature of our work, a robust collaboration workflow was essential. We therefore relied heavily on Git as our version-control system to maintain code integrity, manage feature development, and ensure coordinated teamwork throughout the project. \cite{git_repo}

\subsection{Literature Study}
The structure and content of CircuitQuest are inspired by the lectures and associated textbook used in our previous course on computer architecture:
\begin{quote}
\textit{Rechnerorganisation und -entwurf: Die Hardware / Software-Schnittstelle} \\
David A. Patterson and John L. Hennessy
\end{quote}\cite{referenceBookGerman}
This foundational literature provided essential guidance for establishing the learning progression represented within our tool.
Furthermore, we consulted various online resources and documentation related to digital logic design, Python programming, and the Qt framework to enhance our understanding and implementation of the project.