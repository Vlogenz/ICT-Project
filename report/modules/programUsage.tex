% - How to install
% - How to use everything, Teaching Mode and Sandboxmode
% - Levelprogression, recommend Levels
% - Usage as Customcomponents


\section{Program Usage}
\subsection{Installation and Executing From Source}
% Im going to write this section as the project stands currently, I.E. using a venv

Running the program should be done from a terminal.

\begin{itemize}
    \item For Windows: Use powershell, cd to the project root directory
    \item For Linux: In your preferred terminal, cd to the project root directory
    \item For Mac: Ask Lorenz\\
\end{itemize}

\paragraph{Instructions}

Dependencies:
\begin{itemize}
    \item \texttt{python}, Version 3.13 was used in development, but newer versions will likely work.\\

\end{itemize}

Step 1, setting up the environment:
\begin{itemize}
    \item All: \texttt{python3.13 -m venv env}\\

\end{itemize}

Step 2, entering the environment:
\begin{itemize}
    \item Linux: \texttt{source env/bin/activate}
    \item Windows: \texttt{\& "./env/Scripts/Activate.ps1"}.  If an error comes up saying that running scripts is disabled, 
    run \texttt{Set-ExecutionPolicy -ExecutionPolicy Unrestricted -Scope Process} first.
    \item Mac: \\

\end{itemize}

Step 3, installing/updating packages:
\begin{itemize}
    \item All: \texttt{pip install -r requirements.txt}.\\

\end{itemize}

Step 4, Running:
\begin{itemize}
    \item All: \texttt{python3 -m src.main}.\\

\end{itemize}
